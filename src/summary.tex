\documentclass{beamer}

\usepackage{algorithm}
\usepackage{algorithmic}

\usepackage{amsthm}
\usepackage{amsmath}
\usepackage{amssymb}
\usepackage{amsfonts}

% Theorems
\newtheorem{proposition}{Proposition}
\newtheorem{remark}{Remark}
\newtheorem{exercise}{Exercise}
\newtheorem{hint}{Hint}
\newtheorem{observation}{Observation}
\newtheorem{claim}{Claim}

% Fields
\newcommand{\nn}{\mathbb{N}}
\newcommand{\rr}{\mathbb{R}}

% Circulation
\newcommand{\cl}{\text{c}}

\title{Bipartite Perfect Matching is in RNC}
\author{Alon Gurny}
\date{\today}

\begin{document}

\frame{\titlepage}

% \begin{abstract}
%   This is a summary of
%   \emph{"Bipartite Perfect Matching is in Quasi-NC"}
%   by Fenner, Gurjar and Thierauf.

%   Perfect matching is a fundamental problem in combinatorial
%   optimization and Complexity Theory.

%   We will show an algorithm for perfect matching in bipartite graphs
%   that runs in randomized NC,
%   with only $O(\log^2 n)$ random bits. This is for both the decision
%   and the search (TODO) versions.

%   This is almost a complete de-randomization: it gives time
%   $$O(2^{O(\log^2 n)}) = O(n^{O(\log n)})$$
% \end{abstract}

\begin{frame}
  \frametitle{Outline}
  \tableofcontents
\end{frame}

% \subsection{Notes}
% The notations and definitions in this section are largely taken from the paper.

% However, some notations are slightly different here.
% In particular, we assume a bipartite graph to have $n + n$ vertices instead
% of $n$ vertices.

% \subsection{Combinatorics and Probability}
% In this section, we remind ourselves of results in combinatorics and
% probability that will come up as useful.

\begin{frame}
  \frametitle{Combinatorics and Probability}

  Throughout this presentation, all graphs are
  \emph{undirected, balanced, labeled bipartite graphs}.

  The set of edges can be regarded as a relation $E \subseteq [n] \times [n]$.
\end{frame}

\begin{frame}
  % Definitions
  \begin{definitions}
    \begin{itemize}
      \item The \emph{bi-adjacency matrix}: $A_G = A_E = A$ of $G$/$E$ is an $n \times n$ matrix where
            $A_{ij} = 1_{(i, j) \in E}$
      \item Define $B \le C$ iff $\forall i, j. B_{ij} \le C_{ij}$
      \item If $E$ are edges, for which $B$ does $B \le A_E$ hold?
      \item For which $E$ is $A_E$ a permutation matrix?
    \end{itemize}
  \end{definitions}
\end{frame}
\begin{frame}
  \begin{definitions}
    Note that if $B \in \rr^{n \times n}$ and $E$ a set of edges,
    then $B \le A_E$ if and only if the nonzero entries of
    $B$ are a subset of the edges in $E$.

    A \emph{perfect matching} is a set of edges $M \subseteq E$ such
    that every vertex in $V$ is incident to exactly one edge in $M$.

    A \emph{weight function} is a function $w : E \to \nn$.

    We extend it naturally to a function $w : 2^E \to \nn$ by setting
    $w(S) = \sum_{e \in S} w(e)$ for all $S \subseteq E$.

    Even more generally, we extend it to a function $w : \rr^{n \times
        n} \to \rr$ by setting
    $w(A) = \sum_{i, j} w(A_{ij})$ for all $A \in \rr^{n \times n}$.

    Given $w : E \to \nn$, we define the \emph{weight} of a perfect
  \end{definitions}
\end{frame}

\begin{frame}

  % Definition - the perfect matching polytope
  \begin{definitions}
    The \emph{perfect matching polytope} $P_G$ is the convex hull of
    the bi-adjacency matrices of all the perfect matchings of $G$.
  \end{definitions}

  % Exercise - doubly stochastic matrices are the convex hull of
  % permutation matrices
  \begin{definitions}
    A matrix $M \in \nn^{n \times n}$ is \emph{doubly stochastic} if
    $M$ is entrywise nonnegative and the sum of the entries in each row
    and column is $1$.

    Equivalently, $M$ is doubly stochastic if $M \ge 0$ (entrywise) and
    $M \mathbf{1} = M^T \mathbf{1} = \mathbf{1}$.

    The set of doubly stochastic matrices is denoted by $B_n$ and is
    called the \emph{Birkhoff polytope}.
  \end{definitions}

  \begin{exercise}[Birkhoff's theorem (1946)]
    The set of doubly stochastic matrices is the convex hull of the
    permutation matrices.
  \end{exercise}

  \begin{hint}
    Use Hall's marriage theorem.
  \end{hint}

  \begin{corollary}
    The perfect matching polytope $P_G$ is exactly the matrices $B$ in
    the Birkhoff polytope $B_n$ such that $B \le A_E$.
  \end{corollary}

  \begin{claim}
    The symmetric difference of two perfect matchings is a union of
    disjoint cycles.

    Each cycle consists of interleaved edges from the two matchings.
  \end{claim}

  \begin{proof}
    The degree of each vertex is $1$ in a perfect matching.

    Thus, in the symmetric difference of two perfect matchings, each
    vertex has degree $0$ or $2$.

    Therefore it is a union of disjoint cycles. The rest is easy to see.
  \end{proof}

  % TODO: add examples

  \subsection{Complexity}
  TODO.

  \section{Isolation}

  % Define the Edmonds matrix

  The \emph{weighed Edmonds matrix} Let $w : E \to \nn$ be a weight
  function. $D_{G,w}$ of $G$ is the $n \times n$ matrix where
  \[
    D_{G,w}(i, j) =
    \begin{cases}
      x_{ij}^{w(ij)} & \text{if } (i, j) \in E, \\
      0              & \text{otherwise}.
    \end{cases}
  \]
  for all $i, j \in [n]$.

  \begin{definition}
    A weight function $w : E \to \nn$ is called \emph{isolating} for
    $G$ if $G$ has a unique minimum weight perfect matching.
  \end{definition}

  \begin{lemma}
    If $G$ has a perfect matching and
    if $w : E \to \nn$ has a unique minimum weight perfect matching $M$
    with weight $w(M)$,
    then the term of minimum total degree has weight $w(M)$
    and contains precisely the variables $x_{ij}$ for $(i, j) \in M$.

    If $G$ has no perfect matching, then $\det D_{G,w} = 0$ for all
    weight functions $w$.
  \end{lemma}

  \begin{proof}
    This is obvious by the definition of the determinant.

    \[
      \det D_{G,w} = \sum_{\sigma \in S_n} \text{sgn}(\sigma) \prod_{i
        = 1}^n D_{G,w}(i, \sigma(i))
    \]
  \end{proof}

  \begin{lemma}[Isolation lemma](Mulmuley, Vazirani \& Vazirani 1987)
    Let $B$ be a finite set,
    and let $k$ be a positive integer.
    and let $\mathcal{F} \subseteq 2^B$ be a nonempty family of
    subsets of $B$.

    Let $w : B \to k$ be a be chosen uniformly at random
    among all functions from $B$ to $[k]$.

    Then, with probability at least $1 - |A|/k$,
    there exists a \emph{unique} set $S \in \mathcal{F}$
    with a minimum weight among all sets in $\mathcal{F}$.
  \end{lemma}

  % TODO: rephrase the corollary statement

  \begin{corollary}
    There exists a polynomial-time randomized reduction from
    $\text{PM}$ to $\text{PIT}$ for the determinant of a matrix
    with entries which are powers of variables.

    There exists a polynomial-time randomized reduction from
    $\text{Search-PM}$ to calculating the determinant of such a matrix.
  \end{corollary}

  \begin{proof}
    Let $m$ be the number of edges in $G$.

    The reduction will choose a random weight function $w : [m] \to [m^2]$.
    Let $\mathcal{F}$ be the family of perfect matchings of $G$.

    If $G$ has a perfect matching, then $\mathcal{F}$ is nonempty:
    by the isolation lemma, with probability at least $(1 - m/m^2) = 1 - 1/m =
      \Theta(1 - 1/m)$, $w$ is isolating for $G$.

    In the decision version, this amounts to checking whether the
    determinant of $D_{G,w}$ is nonzero.
    In the search version, this amounts to finding the minimum total
    degree term of the determinant of $D_{G,w}$.
  \end{proof}

  \section{Main Content}

  \begin{definition}
    Let $w : E \to \nn$ be a weight function.
    The \emph{circulation} of a cycle $C = {e_1, \ldots, e_{2k}}$ is defined as
    \[
      \cl_w(C) = |w(e_1) - w(e_2) + \ldots + w(e_{2k - 1}) - w(e_{2k})|
    \]

    This is well defined because we take the absolute value.
  \end{definition}

  \begin{remark}
    Let $M_1$ and $M_2$ be two perfect matchings of $G$,
    and suppose $C\subseteq M_1 \triangle M_2$.

    Then $\cl_w(C) = |w(C \cap M_1) - w(C \cap M_2)|$.
  \end{remark}

  \begin{lemma}
    Suppose $G$ has a perfect matching.
    Let $w : E \to \nn$ be a weight function.

    If every cycle in $G$ has a nonzero circulation, then $w$ is
    isolating for $G$.
  \end{lemma}

  \begin{proof}
    On the contrary, suppose that $M_1$ and $M_2$ are two minimum
    perfect matchings of $G$.
    Choose some cycle $C \subseteq M_1 \triangle M_2$.

    Since $\cl_w(C) > 0$, we have $w(C \cap M_1) \ne w(C \cap M_2)$.
    WLOG, assume that $w(C \cap M_1) < w(C \cap M_2)$.
    Then $M_2 \triangle C$ is a perfect matching with weight less than
    $w(M_2) = w(M_1)$,
    contradicting minimality.
  \end{proof}

  \begin{lemma}
    Let $r \ge 2$ be even.
    Suppose $G$ has no cycles of size at most $2r$.

    Then $G$ has at most $n^4$ cycles of size at most $4r$.
  \end{lemma}

  \begin{proof}
    The assumption is equivalent to saying that there exists at most
    one path of length $r$ between any two vertices.

    Let $v_0, v_1, v_2, v_3$ be vertices in $V_1$, such that the
    distance between $i$ and $v_{i + 1 \mod 4}$ is at most $r$.

    Since there exists at most one path of length $r$ between $i$ and
    $v_{i + 1 \mod 4}$,
    there exists a partial function from $V_1^4$ to cycles in $G$ that maps
    $(v_0, v_1, v_2, v_3)$ to the cycle composed of the unique paths of
    length $r$ between $i$ and $v_{i + 1 \mod 4}$.

    Let $C$ be a cycle of size at most $4r$.

    We can choose arbitrarily $v_0, v_1, v_2, v_3 \in C$ such that
    the distance between $i$ and $v_{i + 1 \mod 4}$ is at most $r$.

    Therefore, this function is onto the set of cycles of size at most $4r$.
    Thus, there exist at most $|V_1|^4 = n^4$ cycles of size at most $4r$.
  \end{proof}

  \begin{lemma}
    Suppose each edge in $H$ is contained in some perfect matching. Let
    $w : E \to \nn$ be a weight function,
    such that the weight of every perfect matching is the same.

    Then the circulation of every cycle in $H$ is zero.
  \end{lemma}

  \begin{proof}
    Let $X$ be the average of all perfect matchings.
    Specifically, let $t$ be the number of perfect matchings, and let
    $A_1, \ldots, A_t$ be the perfect matchings.
    Then $X = \frac{1}{t} \sum_{i = 1}^t M_i$.

    Since each edge is contained in some perfect matching, we have
    $X_{ij} \ge \frac{1}{t}$ for all $i, j$ such that
    $(i, j) \in E$.

    Let $\varepsilon = \frac{1}{t}$.

    Let $C = {e_1, \ldots, e_{2p}}$ be a cycle in $H$.

    Define $Y$ by
    \[
      Y_{ij} =
      \begin{cases}
        X_{ij} + (-1)^{k} \varepsilon & \text{if } (i, j) = e_k \in C, \\
        X_{ij}                        & \text{otherwise}.
      \end{cases}
    \]

    Then clearly $Y \ge 0$ and $Y \mathbf{1} = Y^T \mathbf{1} = \mathbf{1}$.
    Therefore, $Y$ lies in the perfect matching polytope.

    Since all the perfect matchings have the same weight, $w(T) = w(X)$
    and thus $w(Y - X) = 0$.
    But $\cl_w(C) = \varepsilon w(Y - x)$, and thus $\cl_w(C) = 0$.
  \end{proof}

  \begin{lemma}
    Let $s = \text{poly}(n)$. Using $O(\log n)$ random bits we can
    generate a weight assignment
    $w : E \to \nn$ with $|w| = \text{poly}(n)$ such that for every set
    of $s$ cycles,
    $w$ gives nonzero circulation to all of them with probability at
    least $1 - 1/n$.
  \end{lemma}

  \begin{proof}
    Let the cycles be $C_1, \ldots, C_s$. Then, $\cl_w(C_i) \ne 0$ for
    all $i$ is equivalent to
    $\prod_{i = 1}^s \cl_w(C_i) \ne 0$. This product is bounded by
    $\text{poly}(n)^s = 2^{\text{poly}(n)}$.

    Thus, it has at most $\text{poly}(n)$ prime factors, say $k$ prime
    factors. Choose $t = kn$.

    Then if we choose a random prime amongst the first $t$ primes
    $[p_1, \ldots, p_t]$, then if the product is nonzero,
    with probability at least $1 - 1/n$ it is still true modulo the chosen prime.
    Since... TODO.
  \end{proof}
\end{frame}

\end{document}
