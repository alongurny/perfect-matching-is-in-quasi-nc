\documentclass{beamer}

\usepackage{algorithm}
\usepackage{algorithmic}

\usepackage{amsthm}
\usepackage{amsmath}
\usepackage{amssymb}
\usepackage{amsfonts}

\usetheme{Frankfurt}

\addtobeamertemplate{navigation symbols}{}{
    \usebeamerfont{footline}
    \usebeamercolor[fg]{footline}
    \hspace{1em}
    \insertframenumber/\inserttotalframenumber
}

% Theorems
\newtheorem{proposition}{Proposition}
\newtheorem{exercise}{Exercise}
\newtheorem{observation}{Observation}
\newtheorem{claim}{Claim}

% Open Problem
\theoremstyle{remark}
\newtheorem{remark}{Remark}
\newtheorem{open}{Open Problem}
\newtheorem{resolved}{Today}
\newtheorem{hint}{Hint}
\newtheorem{notation}{Notation}

% Fields
\newcommand{\nn}{\mathbb{N}}
\newcommand{\rr}{\mathbb{R}}
\newcommand{\ii}{\rr_{\ge 0}}

% Circulation
\newcommand{\cl}{\text{c}}

% Birkhoff polytope
\newcommand{\birk}{\mathcal{B}}

% Bi-adjacency matrix
\newcommand{\biadj}[1]{\mathbf{A}_{#1}}

% Perfect matching polytope
\newcommand{\pmatch}{\mathcal{P}}

% Convex hull
\newcommand{\ch}[1]{\mathbf{CH}(#1)}

% Set comprehension
\newcommand{\setcomp}[1]{\left\{#1\right\}}

% Sign
\newcommand{\sgn}{\mathsf{sgn}}

% Decision version of perfect matching
\newcommand{\dpm}{\mathsf{PM}}

% Search version of perfect matching
\newcommand{\spm}{\mathsf{SearchPM}}

% Complexity classes
\newcommand{\nc}{{\mathcal{NC}}}

% Asymptotic notation
\newcommand{\OO}{\mathcal{O}}

\title{Bipartite Perfect Matching is in Quasi-NC}
\author{Alon Gurny}
\date{\today}

\begin{document}

\frame{\titlepage}

\begin{frame}
  \begin{block}[Abstract]
    This is a summary of "Bipartite Perfect Matching is in Quasi-NC" by Fenner, Gurjar, and Thierauf.
  \end{block}
  \frametitle{Outline}
  \tableofcontents
\end{frame}

\section{Introduction}

% 

% Notation
\begin{frame}
  \frametitle{Notation}

  \begin{notation}

    Throughout this presentation, \emph{ALL} the graphs are:
    \begin{itemize}
      \item undirected
      \item bipartite
      \item balanced
      \item labeled
    \end{itemize}

    Usually we call our graph $G$ and the set of edges $E$.

    The set of edges can be regarded as a relation $E \subseteq [n] \times [n]$.
  \end{notation}
\end{frame}

% Frame: NC
\begin{frame}
  \frametitle{NC}

  \begin{definition}
    \begin{itemize}
      \item We define $\nc^k$ to be the class of problems that can be solved
            by a polynomial-time uniform family of circuits of polynomial
            size and depth $\OO(\log^k n)$.
      \item Equivalently, $\nc^k$ is the class of problems that can be solved
            by a polynomial number of processors in $\OO(\log^k n)$ time.
      \item \[
              \nc = \bigcup_{k \ge 0} \nc^k
            \]
    \end{itemize}
  \end{definition}

  \begin{fact}
    $\mathsf{DET}$,
    the problem of computing the determinant of an $n \times n$ matrix
    with $\mathsf{poly}(n)$-bounded entries,
    is in $\nc^2$.
  \end{fact}
\end{frame}

% Frame: Perfect Matching
\begin{frame}
  \frametitle{Perfect Matching}

  \begin{definition}
    \begin{itemize}
      \item A \emph{perfect matching} of a graph $G$ is a set of edges
            such that every vertex is incident to exactly one edge.
      \item The \emph{perfect matching problem} is to determine whether
            a graph has a perfect matching, or to find one.
      \item The \emph{decision version} of the perfect matching problem
            is denoted $\dpm$.
      \item The \emph{search version} of the perfect matching problem
            is denoted $\spm$.
    \end{itemize}
  \end{definition}
\end{frame}

% Frame: State of the Art
\begin{frame}
  \frametitle{State of the Art}

  It has already been known that perfect matching (whether decision or search)
  can be solved in randomized $\nc$.

  \begin{open}
    Can perfect matching be solved in $\nc$?
  \end{open}

  \begin{resolved}
    There exist algorithms in:
    \begin{itemize}
      \item $\mathsf{Quasi-}\nc^2$ for $\dpm$
            (that is, with $\OO(n^{\log n})$ processors and
            $\OO(\log^2 n)$ depth)
      \item randomized $\nc^2$ for $\dpm$
            with only $\OO(\log^2 n)$ random bits.
      \item (We will not see this) $\mathsf{Quasi-}\nc^2$ for $\dpm$
            (that is, with $\OO(n^{\log n})$ processors and
            $\OO(\log^2 n)$ depth)
      \item (We will not see this) $\nc^3$ for $\spm$
            with only $\OO(\log^2 n)$ random bits.
    \end{itemize}
  \end{resolved}
\end{frame}

\section{Combinatorics and Probability}

\begin{frame}
  % Definitions
  \begin{definitions}[Bi-adjacency Matrix]
    \begin{itemize}
      \item The \emph{bi-adjacency matrix}: $\biadj{G} = \biadj{E} = A$ of $G$ (or $E$) is an $n \times n$ matrix where
            $A_{ij} = 1_{(i, j) \in E}$
      \item We write $B \le C$ iff $\forall i, j. B_{ij} \le C_{ij}$
      \item A \emph{permutation matrix} is a matrix $\biadj{\Gamma}$
            where $\Gamma = \setcomp{(i, \sigma i) : i \in [n]}$ for some permutation
            $\sigma \in S_n$

    \end{itemize}
  \end{definitions}

  \begin{exercise}
    For which $E \subseteq [n] \times [n]$ is $\biadj{E}$ a permutation matrix?
  \end{exercise}

  \begin{exercise}
    For which $B \in \ii^{n \times n}$ and $E$ does $B \le \biadj{E}$ hold?
  \end{exercise}
\end{frame}

\begin{frame}
  \begin{definitions}[Weight Functions]
    \begin{itemize}
      \item A \emph{weight function} is a function $w : E \to \nn$.

      \item We extend it naturally to a function $w : 2^E \to \nn$ by setting
            $w(S) = \sum_{e \in S} w(e)$ for all $S \subseteq E$.

      \item Even more generally (why?), we extend it to a
            \emph{linear function} $w : \rr^{n \times n} \to \rr$ by setting
            $w(A) = \sum_{ij \in E} w(ij) A_{ij}$ for all $A \in \rr^{n \times n}$.
    \end{itemize}
  \end{definitions}
\end{frame}

\begin{frame}
  \begin{definitions}[Perfect Matching]
    \begin{itemize}
      \item A \emph{perfect matching} is a set of edges $M$ such
            that every vertex is incident to exactly one edge in $M$.
      \item Equivalently, $M$ is a perfect matching iff $\biadj{M}$ is a permutation
            matrix.
    \end{itemize}\end{definitions}
\end{frame}

% Birkhoff's theorem
\begin{frame}[allowframebreaks]
  \frametitle{Birkhoff's Theorem}
  % Exercise - doubly stochastic matrices are the convex hull of
  % permutation matrices
  \begin{definition}[Doubly Stochastic Matrix]
    \begin{itemize}
      \item A matrix $M \in \ii^{n \times n}$ is \emph{doubly stochastic} if
            the sum of the entries in each row
            and column is $1$.

      \item Equivalently, $M$ is doubly stochastic if
            $M \mathbf{1} = M^T \mathbf{1} = \mathbf{1}$.
    \end{itemize}
  \end{definition}

  % Definition of the convex hull
  \begin{definition}[Convex Hull]
    The \emph{convex hull} of a set $S$ is the smallest convex set
    containing $S$, denoted $\ch{S}$.

    It is precisely the set of all convex combinations of elements of $S$,
    i.e., $\ch{S} = \setcomp{\sum_{i = 1}^k \lambda_i x_i : x_i \in S, \lambda_i \ge 0, \sum_{i = 1}^k \lambda_i = 1}$.
  \end{definition}

  \begin{theorem}[Birkhoff, 1946]
    The set of doubly stochastic matrices, denoted $\birk_n$, is the convex hull of the
    permutation matrices.
  \end{theorem}

  \begin{hint}
    Use Hall's marriage theorem.
  \end{hint}
\end{frame}

\begin{frame}
  \frametitle{Perfect Matching Polytope}
  \begin{theorem}[Birkhoff, 1946]
    The set of doubly stochastic matrices is the convex hull of the
    permutation matrices.
  \end{theorem}

  % Definition - the perfect matching polytope
  \begin{definition}[Perfect Matching Polytope]
    The \emph{perfect matching polytope} $\pmatch_G$ is the convex hull of
    the bi-adjacency matrices of all the perfect matchings of $G$.
  \end{definition}

  \begin{corollary}
    The perfect matching polytope $P_G$ is exactly the matrices $B$ in
    the Birkhoff polytope $B_n$ such that $B \le \biadj{G}$:
    \[
      \pmatch_G = \setcomp{B : B \in \birk_n \wedge B \le \biadj{G}}
    \]
  \end{corollary}
\end{frame}

\section{Isolation}
\begin{frame}[allowframebreaks]
  \begin{definition}
    Let $M$ be a perfect matching.
    We define its \emph{sign} $\sgn M = \sgn \sigma$ where $\sigma$
    is the permutation given by $M$.
  \end{definition}
  \begin{definition}
    Let ${\left\{w_i : E \to \nn \right\}}_{i \in [k]}$ be weight functions.

    We define the \emph{weighted Edmonds matrix} $D^{E,w}$ as follows:

    \[
      {\left(D^{E,w}\right)}_{ij} =
      \begin{cases}
        {x_{ij}^{(k)}}^{w_k(ij)} & \text{if } (i, j) \in E, \\
        0            & \text{otherwise}.
      \end{cases}
    \]

    Let $a \in \nn^k$ be a vector of variables.
    We also denote $D_a^{E,w} = D^{E,w}(x_{ij}^{(k)} := a_k)$
  \end{definition}
\end{frame}

\begin{frame}[allowframebreaks]
  \begin{definition}
    A weight function $w : E \to \nn$ is called \emph{isolating} for
    $E$ if the existence of a perfect matching in $E$
    implies that $E$ has a unique minimum weight perfect matching,
    which is then called \emph{isolated}.
  \end{definition}

  \begin{corollary}
    \begin{itemize}
      \item If $E$ has no perfect matchings, then
            $\det D^{E,w} = 0$.
      \item If $w$ is isolating for $E$,
            then $\det D^{E,w} \ne 0 \iff $ there is a perfect matching.
    \end{itemize}
  \end{corollary}

  \begin{example}
    Let $E = \left\{e_k\right\}_k$ be an enumeration of the edges.
    Let $w : E \to \nn$ be defined by
    $w(e_k) = 2^k$. \emph{Every} subset of $E$ has a unique
    weight.

    In particular, $w$ is isolating for $E$.

    Is this enough?
  \end{example}
\end{frame}

\begin{frame}
  \frametitle{Isolation Lemma}
  \begin{lemma}[Mulmuley, Vazirani \& Vazirani 1987]
    Let $B$ be a finite set.

    Let $k$ be a positive integer.

    Let $\mathcal{F} \subseteq 2^B$ be a nonempty family of
    subsets of $B$.

    Let $w : B \to k$ be a be chosen uniformly at random
    among all functions from $B$ to $[k]$.

    Then, with probability at least $1 - |B|/k$,
    there exists a \emph{unique} set $S \in \mathcal{F}$
    with a minimum weight among all sets in $\mathcal{F}$.
  \end{lemma}
\end{frame}
\begin{frame}
  % Corollary for perfect matching
  \begin{corollary}
    Let $w : E \to [n^3]$ be a random weight function.

    Then with probability at least $1 - 1/n$, $w$ is isolating for $G$.
  \end{corollary}

  \begin{proof}
    Take $\mathcal{F}$ as the family of perfect matchings of $G$.


    As, $|E| \le n^2$, by the isolation lemma, with probability at least
    \[
      (1 - |E|/n^3) \ge 1 - n^2/n^3 = 1 - 1/n
    \]
    $w$ is isolating for $G$.
  \end{proof}

  \begin{corollary}
    There exists an $\mathsf{RNC}$ algorithm that can solve $\dpm$,
    which uses $\mathsf{poly}(n)$ random bits.
  \end{corollary}
\end{frame}

\section{Cycles and Circulation}

\begin{frame}
  \frametitle{Circulation}
  \begin{definition}
    Let $w : E \to \nn$ be a weight function.
    The \emph{$w$-circulation} of a cycle $C = {e_1, \ldots, e_{2k}}$ is defined as
    \[
      \cl_w(C) = |w(e_1) - w(e_2) + \ldots + w(e_{2k - 1}) - w(e_{2k})|
    \]

    This is well defined because we take the absolute value.
  \end{definition}

  \begin{remark}
    Let $M_1$ and $M_2$ be two perfect matchings of $G$,
    and suppose $C\subseteq M_1 \triangle M_2$.

    Then $\cl_w(C) = |w(C \cap M_1) - w(C \cap M_2)|$.
  \end{remark}
\end{frame}

\begin{frame}
  \begin{claim}
    The symmetric difference of two perfect matchings is a union of
    disjoint cycles.

    Each cycle consists of interleaved edges from the two matchings.
  \end{claim}

  \begin{proof}
    The degree of each vertex is $1$ in a perfect matching.

    Thus, in the symmetric difference of two perfect matchings, each
    vertex has degree $0$ or $2$.

    Therefore it is a union of disjoint cycles. The rest is easy to see.
  \end{proof}
\end{frame}
\begin{frame}
  \begin{lemma}
    Let $w : E \to \nn$ be a weight function.

    If every cycle in $G$ has a nonzero circulation, then $w$ is
    isolating for $G$.
  \end{lemma}

  \begin{proof}
    On the contrary, suppose that $M_1$ and $M_2$ are two minimum
    perfect matchings of $G$.
    Choose some cycle $C \subseteq M_1 \triangle M_2$.

    Since $\cl_w(C) > 0$, we have $w(C \cap M_1) \ne w(C \cap M_2)$.
    WLOG, assume that $w(C \cap M_1) < w(C \cap M_2)$.
    Then $M_2 \triangle C$ is a perfect matching with weight less than
    $w(M_2) = w(M_1)$,
    contradicting minimality.
  \end{proof}
\end{frame}

\section{The Theorem}

% The theorem
\begin{frame}
  \begin{definition}
    Suppose $G$ has a perfect matching. Let $w : E \to \nn$ be a weight function.

    Then $G_w \subseteq G$ is the union of all minimum weight
    perfect matchings of $G$, relatively to $w$.
  \end{definition}

  \begin{theorem}
    Let $G^0 = G$ be and $G^{i+1} = {\left(G^{i}\right)}_{w_i}$.

    Let $k \ge \log n - 2$ and let $w_0, \ldots, w_k : E \to \nn$ be weight functions
    such that $w_i$ gives nonzero circulation to all the
    cycles of size at most $4 \cdot 2^i$ in $G_{i}$.

    Then the joint weight function $w = (w_0, \ldots, w_k)$ is isolating.
  \end{theorem}
\end{frame}

\begin{frame}[allowframebreaks]
  \begin{lemma}
    Let $r \ge 2$ be even.
    Suppose $G$ has no cycles of size at most $2r$.

    Then $G$ has at most $n^4$ cycles of size at most $4r$.
  \end{lemma}

  \begin{proof}
    \begin{itemize}
      \item Equivalently, there is at most
            one path of length $\le r$ between any two vertices.

      \item Let $v_0, v_1, v_2, v_3$ be vertices in $V_1$, such that the
            distance between $v_i$ and $v_{i + 1 \mod 4}$ is $\le r$.

      \item There exists a \emph{partial function} from $V_1^4$ to cycles in $G$ that maps
            $(v_0, v_1, v_2, v_3)$ to the cycle composed of the unique paths of
            length $\le r$ between $v_i$ and $v_{i + 1 \mod 4}$.

      \item Let $C$ be a cycle of size $\le 4r$.

            We can choose arbitrarily $v_0, v_1, v_2, v_3 \in C$ such that
            the distance between $v_i$ and $v_{i + 1 \mod 4}$ is $\le r$.

      \item This function is onto the set of cycles of size $\le 4r$.
            Thus, there are at most $|V_1|^4 = n^4$ cycles of size $\le 4r$.

    \end{itemize}
  \end{proof}
\end{frame}

\begin{frame}[allowframebreaks]
  \begin{lemma}
    Suppose $G$ has a perfect matching and let
    $w : E \to \nn$ be a weight function.

    Then every perfect matching in $G_w$ has the same weight
    as every minimum weight matching in $G$.
  \end{lemma}

  \begin{proof}
    If $w$ is isolating, $G_w$ is just a single perfect matching.

    Otherwise, suppose the minimum weight for perfect matchings in $G$
    is $m$. Suppose there exist $t$ such matchings and let
    $X$ be their average. Then, $w(X) = m$.

    Let $N$ be a perfect matching in $G_w$.
    Every edge is contained in some minimum weight perfect matching,
    so $X - 1/t N \ge 0$.

    It can easily be seen that the sum of every row and column
    is $(t-1)/t$ and thus the matrix $\frac{t}{t-1} X - \frac{1}{t-1}N$ is
    doubly stochastic.

    Therefore, it lies in $\pmatch_G$ and thus has weight at least $m$.

    That is, \[
      m \le w\left(\frac{t}{t-1} X - \frac{1}{t-1}N\right) =
      m \cdot \frac{t}{t-1} - w(N) \cdot \frac{1}{t-1}
    \]

    From that it follows that $w(N) \le m$.
  \end{proof}
\end{frame}

\begin{frame}[allowframebreaks]
  \begin{lemma}
    Suppose $G$ has a perfect matching and let
    $w : E \to \nn$ be a weight function.

    Then the $w$-circulation of every cycle in $G_w$ is zero.
  \end{lemma}

  \begin{proof}
    Let $X$ be the average of all perfect matchings in $G_w$.

    Let $t$ be the number of perfect matchings, $\varepsilon = 1/t$, and let
    $M_1, \ldots, M_t$ be the perfect matchings.
    Then $X = \varepsilon \sum_{i = 1}^t M_i$.

    Since each edge is contained in some perfect matching, we have
    $X_{ij} \ge \varepsilon$ for all $i, j$ such that
    $(i, j) \in E$.

    Let $C = {e_1, \ldots, e_{2p}}$ be a cycle in $H$.

    Define $Y$ by
    \[
      Y_{ij} =
      \begin{cases}
        X_{ij} + (-1)^{k} \varepsilon & \text{if } (i, j) = e_k \in C, \\
        X_{ij}                        & \text{otherwise}.
      \end{cases}
    \]

    Then clearly $Y \ge 0$ and $Y \mathbf{1} = Y^T \mathbf{1} = \mathbf{1}$.
    Therefore, $Y$ lies in the perfect matching polytope.

    Since all the perfect matchings have the same weight, $w(Y) = w(X)$
    and thus $w(Y - X) = 0$.
    But $\cl_w(C) = \varepsilon w(Y - X)$, and thus $\cl_w(C) = 0$.
  \end{proof}
\end{frame}
\begin{frame}
  \begin{lemma}
    Using $\OO(\log n)$ random bits we can
    generate a weight assignment
    $w : E \to \nn$ with $w \le \mathsf{poly}(n)$ such that for every set
    of $n^4$ cycles,
    $w$ gives nonzero circulation to all of them with probability at
    least $1 - 1/n$.
  \end{lemma}

  \begin{proof}
    Let $w(e_k) = 2^k$.
    Let the cycles be $C_1, \ldots, C_s$. Then, $\cl_w(C_i) \ne 0$ for
    all $i$ is equivalent to
    $\prod_{i = 1}^{n^4} \cl_w(C_i) \ne 0$. This product is bounded by
    $\mathsf{poly}(n)^{n^4} = 2^{\mathsf{poly}(n)}$.

    Thus, it has at most $\mathsf{poly}(n)$ prime factors, say $k$ prime
    factors. Choose $t = kn$.

    Then if we choose a random prime amongst the first $t$ primes
    $[p_1, \ldots, p_t]$, then if the product is nonzero,
    with probability at least $1 - 1/n$ it is still true modulo the chosen prime.
  \end{proof}
\end{frame}

% The algorithm
\begin{frame}[allowframebreaks]
  \begin{theorem}
    Let $G^0 = G$ be and $G^{i+1} = {\left(G^{i}\right)}_{w_i}$.

    Let $k \ge \log n - 2$ and let $w_0, \ldots, w_k : E \to \nn$ be weight functions
    such that for every set of $w_i$ gives nonzero circulation to all the
    cycles of size at most $4 \cdot 2^i$ in $G_{i}$.

    Then the joint weight function $w = (w_0, \ldots, w_k)$ is isolating.
  \end{theorem}

  \begin{proof}
    Since $w_i$ gives nonzero circulation to all the cycles
    in $G^i$, all those cycles do not appear in $G^{i+1}$.

    In particular, $G^k$ has no cycles of size at most $4 \cdot 2^k \ge n$.

    That is, $G^k$ has a unique perfect matching.
    We want to show that it is isolated by $w$.
    This follows from the next lemma.
  \end{proof}

  \begin{lemma}
    Suppose $M_1$ is a matching that appears in $G^i$ but not in $G^{i+1}$ and $M_2$ is a matching
    that appears in $G_{i+1}$.

    Then $w_i(M_2) < w_i(M_1)$.
  \end{lemma}

  \begin{proof}
    Since $M_2$ appears in $G_{i+1}$, it has
    the same weight as any $w_i$-minimum weight matching in $G_i$.
    That is $w_{i}(M_2) \le w_i(M_1)$.

    But if $w_i(M_1) = w_i(M_2)$ held, then $M_1$ would have been in $G^{i+1}$
    by its definition.
  \end{proof}

\end{frame}

% The algorithm
\begin{frame}
  \frametitle{Deciding $\dpm$ in randomized $\nc^2$}
  The following algorithm solves $\dpm$ in randomized $\nc^2$
  with $\OO(\log^2 n)$ random bits.

  % Algorithm
  \begin{algorithm}[H]
    \caption{Decide $\dpm$ (randomized)}
    \begin{algorithmic}
      \STATE Let $k = \log n - 2$.
      \STATE Generate random weight functions $w_0, \ldots, w_k$.
      \STATE Generate random numbers $r_0, \ldots, r_k$.
      \STATE Create $w_0, \ldots, w_k$.
      \STATE Combine $w = (w_0, \ldots, w_k)$.
      \STATE Calculate $\det D_r^{G, w}$.
      \RETURN{$\det D_r^{G, w} \ne 0$}
    \end{algorithmic}
  \end{algorithm}
\end{frame}


% The algorithm
\begin{frame}
  \frametitle{Deciding $\dpm$ in randomized $Quasi-\nc^2$}
  The following algorithm solves $\dpm$ in $\nc^2$
  with $\OO(\log^2 n)$ random bits.

  % Algorithm
  \begin{algorithm}[H]
    \caption{Decide $\dpm$ (deterministic)}
    \begin{algorithmic}
      \STATE Let $k = \log n - 2$.
      \FORALL {Possible weight functions $w_0, \ldots, w_k$ and $r$}
      \STATE Calculate $\det D_r^{G, w}$.
      \IF{$\det D_r^{G, w} \ne 0$}
      \RETURN{True}
      \ENDIF
      \RETURN{False}
      \ENDFOR
    \end{algorithmic}
  \end{algorithm}
\end{frame}

\end{document}
